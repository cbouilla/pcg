\documentclass[preprint]{iacrtrans}

\usepackage[utf8]{inputenc}
\usepackage{amssymb, amsmath, amsfonts, amscd}
\usepackage[T1]{fontenc}
\usepackage{graphicx}
\usepackage{url}
\usepackage{xspace}
\usepackage{subcaption}
\usepackage{algorithm2e}
\usepackage{tikz}
\usepackage{cellspace}
\usepackage{multirow}
%\usepackage{parskip}
\usetikzlibrary{patterns}

%\usepackage[noend]{algpseudocode}

\usepackage[pdftex,bookmarks,bookmarksopen,bookmarksdepth=3]{hyperref}
\hypersetup{colorlinks=true,citecolor=red,linkcolor=red,urlcolor=black}


% knuth-style algos
\newcommand{\slug}{\hbox{\kern1.5pt\vrule width2.5pt height6pt depth1.5pt\kern1.5pt}}
\def\xskip{\hskip 7pt plus 3pt minus 4pt}
\newdimen\algindent
\newif\ifitempar \itempartrue % normally true unless briefly set false
\def\algindentset#1{\setbox0\hbox{{\bf #1.\kern.25em}}\algindent=\wd0\relax}
\def\algbegin #1 #2{\algindentset{#21}\alg #1 #2} % when steps all have 1 digit
\def\aalgbegin #1 #2{\algindentset{#211}\alg #1 #2} % when 10 or more steps
\def\aaalgbegin #1 #2{\algindentset{#2111}\alg #1 #2} % when 10 or more steps
\def\alg#1(#2). {\medbreak % Usage: \algbegin Algorithm A (algname). This...
  \noindent{\bf#1}({\it#2\/}).\xskip\ignorespaces}
\def\algstep#1.{\ifitempar\smallskip\noindent\else\itempartrue
  \hskip-\parindent\fi
  \hbox to\algindent{\bf\hfil #1.\kern.25em}%
  \hangindent=\algindent\hangafter=1\ignorespaces}
% end of borrowed macros


\title{Predicting the PCG PSeudo-Random Number Generator In Practice} 

\author{Charles Bouillaguet\inst{1} \and Julia Sauvage\inst{2} \and Florette Martinez\inst{3}}


\institute{% 
University of Lille, France \\ 
\email{charles.bouillaguet@univ-lille.fr}
\and 
Sorbonne University \\
\email{julia.sauvage@etu.upmc.fr}
\and 
LIP6, CNRS, SU ? \\
\email{florette.martinez@lip6.fr}

}

\begin{document}
\maketitle

\keywords{keywords}

\begin{abstract}
  blablabla
\end{abstract}

\section{Introduction}

\section{The PCG Pseudo-Random Number Generator Family}

PCG64 is the default Pseudo-Random Number Generator in the latest version of \textsf{NumPy} (1.18)

% import numpy
% print(numpy.random.default_rng())
% ---> Generator(PCG64)

\section{Warm-Up: Prediction with Known Increment }

[insert here] attack predicting algo description.

\begin{theorem}
  The algorithm described above works with probability.... in time ...
\end{theorem}

\section{Prediction with Unknown Increment}

[insert here] attack predicting algo description. Phase 1, Phase 2, Phase 3

\begin{theorem}
  The algorithm described above works with probability.... in time ...
\end{theorem}

\section{Implementation of the Predictor and Practical Results}

\section{Conclusion}

\bibliographystyle{alpha}
\bibliography{biblio}

%\appendix

\end{document}